\documentclass{beamer}
\usepackage{mathspec,xunicode,xltxtra,graphicx,booktabs,siunitx}
\sisetup{range-phrase = --}

\usepackage{tikz}
\usetikzlibrary{arrows,decorations,fit,shadows,positioning,shapes}

%%% Pacchetti di uso comune per lingua, encoding di input e di font
\usepackage{polyglossia}
\setdefaultlanguage{italian}

\setsansfont[Mapping=tex-text,Ligatures=Common,Numbers={Lining,Proportional}]{Ubuntu}
\setmonofont[Mapping=tex-text,Scale=0.9]{Andale Mono}

%%% Tema, stile di uncover, default overlay e colore di struttura
\usetheme{Warsaw}
\setbeamercovered{transparent}
%\beamerdefaultoverlayspecification{<+->}
\setbeamercolor{alerted text}{fg=structure}
\definecolor{ubuntugray}{HTML}{333333}
\definecolor{ubuntuorange}{HTML}{DD4818}
\setbeamercolor{structure}{bg=ubuntugray,fg=ubuntuorange}
\setbeamercolor{author in head/foot}{bg=ubuntugray}
\setbeamercolor{subsection in head/foot}{bg=ubuntuorange}
\setbeamercolor{section in head/foot}{bg=ubuntugray}

%%% Definisco un po' di modifiche sulla bibliografia
\setbeamertemplate{bibliography item}[triangle]
%\setbeamercolor*{bibliography entry author}{fg=black}

%%% Sovrascrive la definizione della footline per ottenerne una a due colonne
\setbeamertemplate{footline}{%
  \leavevmode%
  \hbox{%
  \begin{beamercolorbox}[wd=.5\paperwidth,ht=2.25ex,dp=1ex,right]{author in head/foot}%
    \usebeamerfont{author in head/foot}\insertshortauthor~~(\insertshortinstitute)\hspace*{2ex}
  \end{beamercolorbox}%
  \begin{beamercolorbox}[wd=.5\paperwidth,ht=2.25ex,dp=1ex]{title in head/foot}%
    \hspace*{2ex}\usebeamerfont{title in head/foot}\insertshorttitle\hfill%
    \insertframenumber{} / \inserttotalframenumber\hspace*{2ex}
  \end{beamercolorbox}}
   \vskip0pt%
}

%%% Definisco shell (mostra console utente) e shell* (console di root)
\setbeamercolor{shell snippet}{fg=lightgray,bg=black}
\newenvironment{shell}{\par\vspace{.5em}\begin{beamercolorbox}[rounded=true,sep=.2em]{shell snippet}\ttfamily {\color{blue}\$}}{\end{beamercolorbox}}
  \newenvironment{shell*}{\par\vspace{.5em}\begin{beamercolorbox}[rounded=true,sep=.2em]{shell snippet}\ttfamily {\color{red}\#}}{\end{beamercolorbox}}
\newcommand{\urlify}[1]{{\color{blue}\url{#1}}}

%%% Definisco snippet
\newenvironment{snippet}[1]{\begin{block}{Estratto: \texttt{#1}}\ttfamily}{\end{block}}

%%% Definisco param per migliorare la leggibilità dei parametri
\newcommand{\param}[1]{$\langle$#1$\rangle$}

%%% Definisco un tema per le definizioni con monospace
\newenvironment{codedesc}
  {\begin{description}\setbeamerfont{description item}{family=\ttfamily}}
  {\end{description}}


%%%% Struttura principale della presentazione %%%%

%%% Global structure %%%  
\title{Gestione del sistema e servizi}
\subtitle{Modulo 3}

\institute[BGlug]{\begin{minipage}{3cm}\centering
  \includegraphics[width=3cm]{logo}
\end{minipage}\quad\begin{minipage}{.5\textwidth}
  Bergamo Linux Users Group\\
  Circoscrizione n° 2, Largo Röntgen n° 3\\
  Bergamo
\end{minipage}}

\date[18/01/2012]{18 gennaio 2012 --- Pausa didattica a.s.~2011/2012\\
ITIS «Paleocapa», Bergamo}

%%% Dovrò specificare un logo...
% \pgfdeclareimage[height=0.5cm]{university-logo}{university-logo-filename}
% \logo{\pgfuseimage{university-logo}}
\logo{\includegraphics[width=2.5cm]{logo}}

\begin{document}

\begin{frame}
  \titlepage
\end{frame}

\begin{frame}
  \frametitle{Indice}
  \begin{columns}
    \begin{column}{.5\textwidth}
      \tableofcontents[sections={1-2}]
    \end{column}
    \begin{column}{.5\textwidth}
      \tableofcontents[sections={3}]
    \end{column}
  \end{columns}
\end{frame}

\section{Dischi, partizioni, filesystems}
\subsection{Dispositivi fisici}
\begin{frame}[t]
  \frametitle{Nomenclatura dischi}

  \begin{block}{Schema generale}\ttfamily
    /dev/\alert{\param{t}}d\alert{\param{a}}
  \end{block}

  \smallskip\onslide<2->
  Esempi:
  \begin{block}{IDE (PATA)}\ttfamily
    /dev/hda, /dev/hdb, /dev/hdc \ldots
  \end{block}
  \smallskip
  \begin{block}{SATA e SCSI}\ttfamily
    /dev/sda, /dev/sdb, /dev/sdc \ldots
  \end{block}

\end{frame}

\begin{frame}[t]
  \frametitle{Nomenclatura partizioni}
  \begin{block}{Schema generale}\ttfamily
    /dev/\alert{\param{t}}d\alert{\param{a}\param{n}}
  \end{block}
  \begin{block}{Tipo e numero}
    Partizioni primarie: $1 \le n \le 4$\\
    Partizioni logiche: $ n \ge 5$
  \end{block}
  \onslide<3->
  Esempi:
  \begin{block}{Sul primo disco SATA}\ttfamily
    /dev/sda1, /dev/sda2, /dev/sda5, /dev/sda6
  \end{block}

\end{frame}

\subsection{Dispositivi logici}

\begin{frame}[t]
  \frametitle{RAID --- Redundant Array of Indipendent Disks}

  \begin{block}{Risultato}
    Uno o più dischi logici (costituiti da più dischi fisici), con certe
    caratteristiche di ridondanza e \textit{fault tolerance}.
  \end{block}

    \begin{itemize}[<+->]
      \item JBOD: \textit{Just a Bunch Of Disks}
      \item RAID 0: \textit{striping}
      \item RAID 1: \textit{mirroring}
      \item RAID 5
      \item RAID 1+0 (10): \textit{striping} su più array di dischi in
	\textit{mirroring}
    \end{itemize}

\end{frame}

\begin{frame}[t]
  \frametitle{LVM --- Logical Volume Management}

  \begin{block}{Risultato}
    Uno o più partizioni logiche le cui dimensioni non sono limitate dalla
    dimensione fisica dei dischi sottostanti, ridimensionabili a piacimento.
  \end{block}

    \begin{itemize}[<+->]
      \item \textit{Physical Volume} (PV)
      \item \textit{Volume Group} (VG)
      \item \textit{Logical Volume} (LV)
    \end{itemize}
\end{frame}

\subsection{Principali filesystem Unix}

\begin{frame}
  \frametitle{Filesystems --- \texttt{ext2}}

  \begin{itemize}[<+->]
    \item Nato nel 1993 e supportato dalla versione di Linux 0.96c
    \item Non è \textit{journaling}
    \item Raccomandato ancora per supporti flash e dischi SSD
    \item Utilizzato spesso per le partizioni di boot per la sua longevità
  \end{itemize}

\end{frame}

\begin{frame}
  \frametitle{Filesystems --- \texttt{ext3}}
  
  \begin{itemize}[<+->]
    \item Nato nel 1998 e incluso nel kernel dalla versione 2.4.15 (2001)
    \item \textit{Journaling} (no checksum!)
    \item Ingrandimento del filesystem \textit{online}
    \item Fino a \num{32000} sottocartelle
    \item Permette l'aggiornamento \textit{in place} da \texttt{ext2} 
  \end{itemize}
\end{frame}

\begin{frame}
  \frametitle{Limiti di \texttt{ext2} e \texttt{ext3}}

  \begin{center}
    \begin{tabular}{ccc}
      \toprule
      \textbf{Dim.~blocco} & \textbf{Dim.~max file} & \textbf{Dim.~max FS}\\
      \midrule
      \SI{1}{\kibi\byte} & \SI{16}{\gibi\byte} & \SI{2}{\tebi\byte}\\
      \SI{2}{\kibi\byte} & \SI{256}{\gibi\byte} & \SI{8}{\tebi\byte}\\
      \SI{4}{\kibi\byte} & \SI{2}{\tebi\byte} & \SI{16}{\tebi\byte}\\
      \SI{8}{\kibi\byte} & \SI{2}{\tebi\byte} & \SI{32}{\tebi\byte}\\
      \bottomrule
    \end{tabular}
  \end{center}

\end{frame}

\begin{frame}
  \frametitle{Filesystems --- \texttt{ext4}}

  \begin{itemize}[<+->]
    \item Ideato nel 2003 ma sviluppato dal 2006, presente dal kernel 2.6.28
    \item \textit{Journaling} (con checksum)
    \item Supporto a filesystem larghi (file max.~\SI{16}{\tebi\byte}, FS
      max.~\SI{1}{\exbi\byte})
    \item Fino a \num{64000} sottocartelle
    \item Utilizzato in Android 2.3 \textit{Gingerbread} e in molte delle
      distribuzioni moderne
    \item Compatibile all'indietro con \texttt{ext2} e \texttt{ext3},
      compatibile in avanti (con limiti) con \texttt{ext3}
  \end{itemize}

\end{frame}

\begin{frame}
  \frametitle{Filesystems --- \texttt{btrfs}}

  \begin{itemize}[<+->]
    \item Sviluppato dal 2007 (Oracle), presente dal kernel 2.6.30
    \item Deframmentazione \textit{online}
    \item Dimensione massima file: \SI{16}{\exbi\byte}
    \item Snapshot e sottovolumi
    \item Checksum su data e metadata
    \item Compressione trasparente
    \item RAID 0, 1 e 1+0 a livello di filesystem
    \item \textit{Scrub} in background
  \end{itemize}

\end{frame}

\begin{frame}
  \frametitle{Limitazioni di \texttt{btrfs}}
  
  \begin{itemize}[<+->]
    \item Non esiste ancora un tool per il check del filesystem
      (\texttt{btrfsck}; supporto sperimentale)
    \item Limite sul numero di hardlink nella stessa directory:
      \numrange{300}{600} link
  \end{itemize}

\end{frame}

\subsection{Struttura delle directory}

\begin{frame}
  \frametitle{Directory secondo FHS --- 1}

  \begin{block}{FHS}
    Filesystem Hierarchy Standard
  \end{block}

  \begin{codedesc}
    \setbeamerfont{description item}{family=\ttfamily}
    \item[/] Directory radice (prima gerarchia)
    \item[/root/] Home dell'utente \alert{root}
    \item[/bin/] Eseguibili necessari in \textit{single user mode}
    \item[/sbin/] Eseguibili necessari per il sistema
    \item[/lib/] Librerie essenziali per gli eseguibili in
      /bin e \texttt{/sbin}
    \item[/boot/] File per i bootloader
    \item[/dev/] Device
    \item[/etc/] File di configurazione
    \item[/home/] Home degli utenti
  \end{codedesc}
  
\end{frame}

\begin{frame}
  \frametitle{Directory secondo FHS --- 2}
  
  \begin{codedesc}
    \item[/media/] Punti di mount per device rimovibili
    \item[/mnt/] Filesystem \emph{temporanei}
    \item[/opt/] Software opzionali (non \textit{opensource})
    \item[/srv/] Dati serviti dal sistema
    \item[/tmp/] File temporanei (non conservati)
    \item[/proc/] Filesystem virtuale per i processi
    \item[/sys/] Filesystem virtuale per il sistema
  \end{codedesc}

\end{frame}

\begin{frame}
  \frametitle{Directory secondo FHS --- 3 (\texttt{/usr})}
  \begin{codedesc}
    \item[/usr/] Programmi e dati per utenti e servizi (seconda
      gerarchia)
    \item[/usr/bin/] Programmi non necessari al sistema (multiutenti)
    \item[/usr/sbin/] Servizi non necessari al sistema
    \item[/usr/lib/] Librerie per i programmi in \texttt{/usr/bin}, \texttt{/usr/sbin}
    \item[/usr/shared/] File invariabili, identici su architetture diverse
    \item[/usr/local/] Programmi e dati locali, spesso compilati (terza gerarchia)
    \item[/usr/include/] \textit{Include file standard}
    \item[/usr/src/] Codici sorgenti (es.~kernel)
  \end{codedesc}
\end{frame}

\begin{frame}
  \frametitle{Directory secondo FHS --- 4 (\texttt{/var})}
  
  \begin{description}
    \item[\texttt{/var/}] File variabili
    \item[\texttt{/var/tmp/}] File temporanei, conservati
    \item[\texttt{/var/cache/}] Dati in cache
    \item[\texttt{/var/lib/}] Dati persistenti modificati da programmi
      (es.~database)
    \item[\texttt{/var/log/}] File di log
    \item[\texttt{/var/lock/}] File di lock
    \item[\texttt{/var/run/}] Informazioni sul sistema dall'ultimo avvio
  \end{description}
\end{frame}

\section{Introduzione alla CLI}
\subsection{Terminali e console}

\begin{frame}[t]
  \frametitle{Terminali e console}\small
  \begin{block}{Console}
    È un sistema di input e visualizzazione per i messaggi di amministrazione
    del sistema; è solitamente costituita da un monitor e da una tastiera
  \end{block}
  \onslide<2->
  In ogni sistema Linux ci sono almeno quattro console virtuali, accessibili
  con le combinazioni di tasti: \texttt{Ctrl + Alt + F\alert{\param{n}}}
  \onslide<3->
  \begin{block}{Terminale (emulatore di terminale)}
    Permette di emulare un videoterminale su altre architetture grafiche
  \end{block}

  \onslide<4>
  In generale, entrambe le modalità di controllo a linea di comando vi
  offriranno la \alert{shell di login}

\end{frame}

\subsection{Interfacce utente}

\begin{frame}
  \frametitle{La shell}

  \begin{block}{Shell}
    Software che fornisce un'interfaccia agli utenti di un sistema operativo
    per utilizzare i servizi forniti dal kernel
  \end{block}

  \onslide<2->
  Una shell può essere sia una CLI (\textit{Command Line Interface}) o una GUI
  (\textit{Graphical User Interface})

  \medskip
  \onslide<3->
  La shell è solitamente \alert{\texttt{bash}}, nel caso di Ubuntu la GUI è
  \alert{GNOME + Unity} 
  
\end{frame}

\begin{frame}[t]
  \frametitle{Primi passi con \texttt{bash}}

  \begin{block}{Comandi di base}
    \begin{itemize}[<+->]
      \item \texttt{ls}
      \item \texttt{touch}
      \item \texttt{cp}
      \item \texttt{mv}
      \item \texttt{rm}
      \item \texttt{cd}
      \item \texttt{mkdir}
      \item \texttt{rmdir}
    \end{itemize}
  \end{block}

\end{frame}

\begin{frame}
  \frametitle{Aiuto! Non so cosa sto facendo!}

  \begin{block}{Comandi di aiuto}
    \begin{itemize}[<+->]
      \item \texttt{apropos}
      \item \texttt{man}
      \item \texttt{info}
      \item opzioni \texttt{-h} e \texttt{-{}-help}
    \end{itemize}
  \end{block}
  
\end{frame}

\section{Sistema e servizi}

\subsection{Avvio del sistema}

\begin{frame}
  \frametitle{Avvio di un sistema Linux}
  \begin{center}
  \begin{tikzpicture}[every node/.style={rounded
    corners,rectangle,drop shadow,fill=ubuntuorange,draw=ubuntugray,text=white},
    ->,>=stealth,auto,every shadow/.style={fill=ubuntugray,opacity=0.75}]
    \node (bootloader) {Bootloader};
    \onslide<2->
    \node (kernel)  [below=of bootloader] {Kernel} edge[<-] (bootloader);
    \onslide<3->
    \node[ellipse] (upstart)  [below=of kernel] {Upstart} edge[<-] (kernel);
    \onslide<4->
    \node[ellipse] (sysv)  [below=of upstart] {SysV \texttt{rc}} edge[<-] (upstart);
  \end{tikzpicture}
  \end{center}
\end{frame}

\subsection{Gestione dei servizi}

\begin{frame}
  \frametitle{Ordine di avvio e arresto --- Upstart}

  \begin{snippet}{/etc/init/cron.conf}
    start on runlevel [2345]\\
    stop on runlevel [!2345]
  \end{snippet}
\end{frame}

\begin{frame}
  \frametitle{Ordine di avvio e arresto --- SysV}
  Basato su directory per runlevel (\texttt{/etc/rc\alert{\param{n}}.d})
  al cui interno ci sono alcuni \alert{link simbolici}:\\
  \texttt{\alert{K20}virtualbox-guest-utils}\\
  \texttt{\alert{S25}bluetooth}
  
  \medskip
  I link simbolici puntano a script in \texttt{/etc/init.d}
\end{frame}

\begin{frame}[t]
  \frametitle{Avvio, riavvio e arresto --- CLI/Upstart}
  
  \begin{itemize}[<1->]
    \item Avvio:
      \begin{shell*}
        start \alert{\param{servizio}}
      \end{shell*}
    \item Riavvio:
      \begin{shell*}
        restart \alert{\param{servizio}}
      \end{shell*}
    \item Arresto:
      \begin{shell*}
        stop \alert{\param{servizio}}
      \end{shell*}
  \end{itemize}
\end{frame}

\begin{frame}[t]
  \frametitle{Avvio, riavvio e arresto --- CLI/SysV}
  \begin{itemize}[<1->]
    \item Avvio:
      \begin{shell*}
        service \alert{\param{servizio}} start
      \end{shell*}
    \item Riavvio:
      \begin{shell*}
        service \alert{\param{servizio}} restart
      \end{shell*}
    \item Arresto:
      \begin{shell*}
        service \alert{\param{servizio}} stop
      \end{shell*}
  \end{itemize}
\end{frame}

\begin{frame}[t]
  \frametitle{Avvio, riavvio e arresto --- GUI}
  
  \begin{block}{Upstart}
    \texttt{jobs-admin}
    \begin{shell}
      sudo apt-get install jobs-admin
    \end{shell}
  \end{block}
  \onslide<1->
  \begin{block}{SysV}
    \texttt{bum} --- Boot Up Manager
    \begin{shell}
      sudo apt-get install bum
    \end{shell}
  \end{block}
\end{frame}

\subsection{Servizi principali}

\begin{frame}
  \frametitle{Servizi principali --- Generale}

  \begin{codedesc}
  \item[cron] Esegue task programmati (\texttt{anacron}, \texttt{crond},
    \texttt{atd})
  \item[syslog] Logger di sistema (raccoglie messaggi dal sistema)
    (\texttt{rsyslog})
  \item[acpid] Gestione di eventi hardware
  \item[dbus] \textit{Message Bus System}
  \item[udev] Gestore dei device (hotplug, coldplug)
  \item[networking] Gestione delle interfacce di rete
  \item[hwclock] Gestione dell'orologio hardware del sistema
  \item[ufw] Ubuntu Firewall (disabilitato)
  \end{codedesc}
\end{frame}

\begin{frame}
  \frametitle{Servizi principali --- Desktop}
  
  \begin{codedesc}
  \item[pulseaudio] Gestione dell'audio di sistema
  \item[bluetooth] Gestione dei dispositivi Bluetooth
  \item[cups] Server di stampa
  \item[saned] Server per le scansioni
  \end{codedesc}
\end{frame}

\begin{frame}[t]
  \frametitle{Servizi principali --- Server}
  
  \begin{codedesc}
  \item[apache2] Server Web
  \item[mysql] Database server MySQL
  \item[ssh] Server \textit{Secure SHell} (SSH)
  \item[openntpd] Server per la sincronizzazione dell'ora
  \item[slapd] Server di directory LDAP
  \item[squid] Proxy Server
  \item[samba] Condivisione file \textit{Short Message Block} (SMB, usato da
    client Windows)
  \item[nfs-common] Condivisione file \textit{Network File System} (NFS)
  \end{codedesc}
\end{frame}

\begin{frame}[t]
  \frametitle{Sicurezza e servizi}

  \begin{block}{Avvertimento}
    Un servizio attivo ma che non è in uso è un potenziale pericolo per la
    sicurezza del sistema, soprattutto se è in ascolto su porte TCP o UDP!
  \end{block}

  \onslide<2->
  \begin{block}{Consiglio}
    Potete sempre verificare le porte aperte con \texttt{netstat}:
    \begin{shell}
      netstat -lntu
    \end{shell}
  \end{block}
  
\end{frame}

\begin{frame}[t]
  \frametitle{Disabilitare l'avvio --- Upstart}

  \begin{enumerate}[<+->]
    \item Usare il file override:
      \begin{shell*}
        echo 'manual' > /etc/init/\alert{\param{servizio}}.override
      \end{shell*}
    \item Usare il file di configurazione (fare backup del vecchio conf!)
      \begin{shell*}
        mv /etc/init/\alert{\param{servizio}}.conf\{,.bak\}; \textbackslash\\
        \quad echo 'manual' > /etc/init/\alert{\param{servizio}}.conf
      \end{shell*}
  \end{enumerate}
\end{frame}

\begin{frame}
  \frametitle{Disabilitare l'avvio --- SysV}
  
  \begin{shell*}
    update-rc.d -f \alert{\param{servizio}} remove
  \end{shell*}
\end{frame}

\subsection{Permessi su file e directory}

\begin{frame}[t]
  \frametitle{Permessi sui file e directory}

  \begin{shell}\small
    ls -lh\\
    total 0\\
    \alert<4>{drwxr-xr-x}	2 \alert<2>{ubuntu}   \alert<3>{ubuntu}
    {} 68B
    17 Gen 18:00  directory\\
    \alert<4>{-rw-r-{}-r-{}-}	1 \alert<2>{ubuntu}   \alert<3>{ubuntu}  464B
    17 Gen 18:00  file.txt
  \end{shell}

  \only<2>{%
  \begin{block}{Utente proprietario}
    Tipicamente chi ha creato il file/directory.
  \end{block}}

  \only<3>{%
  \begin{block}{Gruppo proprietario}
    Solitamente il gruppo principale del proprietario. Viene usato per
    concedere a più utenti permessi più ampi.
  \end{block}}

  \only<4>{%
  \begin{block}{Permessi}
    Da leggersi a gruppi di tre lettere e nell'ordine \texttt{r} (lettura,
    \textit{read}), \texttt{w} (scrittura, \textit{write}) e \texttt{x}
    (esecuzione, \textit{execute}).
  \end{block}}
\end{frame}

\begin{frame}[t]
  \frametitle{Modifica permessi: \texttt{chmod} --- Rappr.~simbolica}

  \begin{block}{Codici per gli utenti}
    \begin{center}
      \begin{tabular}{l>{\ttfamily}c}
	\toprule
	\textbf{Utente} & \textsf{\bfseries Sigla}\\
	\midrule
	Utente proprietario (\textit{user}) & u\\
	Gruppo proprietario (\textit{group}) & g\\
	Altri utenti (\textit{others}) & o\\
	Tutti i precedenti (\textit{all}) & a\\
	\bottomrule
      \end{tabular}
    \end{center}
  \end{block}

  \onslide<2->
  \begin{shell*}
    chmod a+r u+w file.txt
  \end{shell*}
\end{frame}

\begin{frame}
  \frametitle{Modifica permessi: \texttt{chmod} --- Rappr.~ottale}

  \begin{block}{Codici letterali ed ottali}
    \begin{center}
      \begin{tabular}{l>{\ttfamily}c>{\ttfamily}c}
	\toprule
	\textbf{Permesso} & \textsf{\bfseries Cod.~lett.} &
	\textsf{\bfseries Cod.ottale}\\
	\midrule
	read & r & 4\\
	write & w & 2\\
	execute & x & 1\\
	\bottomrule
      \end{tabular}
    \end{center}
  \end{block}

  \onslide<2->
  \begin{shell*}
    chmod 644 file.txt
  \end{shell*}
\end{frame}

\begin{frame}[t]
  \frametitle{Esempi «sensati» di modifiche}
  \begin{shell*}
    chmod 664 file.txt
  \end{shell*}

  \begin{shell*}
    chmod +x script.sh
  \end{shell*}

  \begin{shell*}
    chmod a+rx ug+w directory
  \end{shell*}

  \begin{shell*}
    chmod -R g+w directory
  \end{shell*}

  \begin{shell*}
    chmod 751 directory
  \end{shell*}

\end{frame}

\begin{frame}
  \frametitle{Modificare i proprietari --- \texttt{chown}}

  \begin{shell*}
    chown \alert<2>{ubuntu}:\alert<3>{ubuntu} file.txt
  \end{shell*}
  
\end{frame}

\begin{frame}
  \frametitle{Modificare permessi e proprietari da GUI}
  
  Vediamone un esempio \textit{live}.
\end{frame}

\subsection{Gestione dei processi e del sistema}

\begin{frame}[t]
  \frametitle{Processi e sistema --- CLI}
  
  \begin{block}{Comandi principali}
    \begin{itemize}
      \item \texttt{top}
      \item \texttt{ps} --- Lista dei processi
      \item \texttt{free} --- Utilizzo della memoria RAM/swap
      \item \texttt{df} e \texttt{du} --- Utilizzo del disco
    \end{itemize}
  \end{block}

  \begin{block}{Filesystem virtuali}
    \begin{itemize}
      \item \texttt{/proc}, procfs --- Processi
      \item \texttt{/sys}, sysfs --- Sistema
    \end{itemize}
  \end{block}
\end{frame}

\begin{frame}
  \frametitle{Processi e sistema --- GUI}
  Vediamo \textit{Monitoraggio sistema} e \textit{Utilizzo disco}.
\end{frame}

\subsection{Gestione remota}

\begin{frame}[t]
  \frametitle{Gestione remota con \texttt{sshd}}
  \begin{block}{Installazione del server}
    \begin{shell}
      sudo apt-get install openssh-server
    \end{shell}
  \end{block}

  \onslide<2->
  \begin{block}{Uso del client}
    \begin{shell}
      ssh \alert{\param{username}}@\alert{\param{hostname o indirizzo IP}}
    \end{shell}
  \end{block}
  La porta standard è la 22

\end{frame}

\begin{frame}
  \frametitle{\texttt{sshd} e \texttt{screen}}

    \begin{shell}
      screen
    \end{shell}
    Quindi, potete \textit{staccare} la sessione di \texttt{screen} con
    \texttt{Ctrl + A} e poi \texttt{d}.

    \onslide<2->
    Si può \textit{riattaccare} la sessione di \texttt{screen} utilizzando:
    \begin{shell}
      screen -r
    \end{shell}
    In caso di più sessioni aperte, lo stesso \texttt{screen} suggerirà quali
    sono aperte e da che data e ora.
\end{frame}

\end{document}
